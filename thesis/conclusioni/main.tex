\chapter{Conclusioni}

Sono state implementate le regole di inferenza che costituiscono il
type system per il $\pi$-calcolo; per fare questo, \`e stato necessario
fornire una rappresentazione per tutti i concetti matematici coinvolti.

Osservando il codice, o semplicemente l'indice di questa tesi, ci si
rende conto immediatamente di come l'implementazione vera e propria
delle regole di inferenza sia stata di fatto una parte minima del lavoro,
mentre la quasi totalit\`a degli sforzi \`e stata necessariamente
concentrata sulla definizione dei vari tipi di dato e delle procedure
che li manipolano.

Un'altra cosa che si pu\`o osservare \`e che le righe di codice scritte
per i casi di test sono quasi il doppio rispetto a quelle scritte per
l'implementazione della libreria: gli oltre 220 casi di test sviluppati
nel corso del progetto hanno consentito di acquisire una buona confidenza
nella correttezza dell'implementazione, oltre che permettere di
verificare che ogni cambiamento fatto non causasse conseguenze
inaspettate.

Il linguaggio Racket, con la sua concisione e naturale inclinazione alla
rappresentazione di concetti matematici, ha consentito di mantenere la
complessit\`a delle singole unit\`a computazionali sotto controllo: come
si \`e visto, la procedura pi\`u lunga conta circa 30 righe di codice,
mentre la maggior parte \`e costituita da meno di 10.

Il progetto, essendo stato sviluppato sotto forma di libreria, pu\`o
essere integrato comodamente in qualsiasi altro software scritto in
Racket; inoltre, grazie alla rappresentazione testuale fornita per i
vari tipi di dato, l'integrazione con software scritti in altri linguaggi
non presenta particolari ostacoli: sarebbe semplice scrivere un piccolo
server che si comporta come il toplevel presentato, e che comunica con
un altro programma, magari scritto in C, tramite una socket.

Esistono possibilit\`a di estensione del progetto al di fuori della sua
integrazione con altri software: ad esempio, sarebbe possibile
estendere Pity implementando la conversione da processi poliadici a
processi monadici ed il type system per il $\pi$-calcolo monadico
descritti in \cite{qw05}.
