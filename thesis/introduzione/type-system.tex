\subsection{Type system per il \texorpdfstring{$\pi$}{pi}-calcolo}

Per garantire che le comunicazioni tra i processi avvengano in maniera
corretta, utilizziamo un type system basato su regole di
inferenza\cite{qw05}.

Prima di introdurre le regole di inferenza che costituiscono il type system
vero e proprio, \`e necessario spiegare i concetti di \emph{sort} e
\emph{sorting}, indispensabili alla sua realizzazione.

Il concetto di sort nel $\pi$-calcolo \`e in qualche modo equivalente al
concetto di tipo in un linguaggio di programmazione: il sort assegnato ad un
nome determina sia quali altri nomi possano essere trasmessi o ricevuti usando
quel nome come canale di comunicazione, sia quali altri nomi possano
trasmettere o ricevere il nome in questione.

L'insieme dei sort \`e $\mathcal{S}$, e un sorting \`e una funzione
$\lambda : \mathcal{S} \rightarrow \mathcal{S}^{+}$ che assegna ad ogni sort
una lista non vuota di sort: un esempio di sorting pu\`o essere

\begin{pilisting}
$
    \lambda(s) = (t,r) \;\;\;
    \lambda(t) = (s)   \;\;\;
    \lambda(r) = (r)   \;\;\;\;\;\;
$
con $\mathcal{S} = \{s,t,r\}$
\end{pilisting}

Questo sorting stabilisce che un nome di sort $s$ possa trasmettere due
nomi, rispettivamente di sort $t$ e $r$; allo stesso modo, si pu\`o vedere
come un nome di sort $r$ possa essere trasmesso o come secondo nome su un
canale di sort $s$, sia come unico nome su un canale di sorting $r$.

Dato un processo $P$, la corrispondenza tra i vari nomi $x_i \in n(P)$ e
i rispettivi sort $s_i \in \mathcal{S}$ viene rappresentata da un
\emph{ambiente} $\Psi$.

Possiamo finalmente elencare le regole di inferenza che costituiscono il
type system:

\begin{pilisting}
\inference[nil ]
  {}
  {\Psi |- 0}

\vspace{3mm}
\indent
\inference[rep ]
  {\Psi |- P}
  {\Psi |- !P}

\vspace{3mm}
\indent
\inference[par ]
  {\Psi |- P_1 & \Psi |- P_2}
  {\Psi |- P_1|P_2}

\vspace{3mm}
\indent
\inference[res ]
  {\Psi,z:s |- P}
  {\Psi |- (z)P}
$\; (1)$

\vspace{3mm}
\indent
\inference[out ]
  {\Psi |- P}
  {\Psi,x:s,y_1:t_1,\ldots,y_n:t_n |- \overline{x}\langle y_1\ldots y_n\rangle.P}
$\; (2)$

\vspace{3mm}
\indent
\inference[inp ]
  {\Psi,z_1:t_1,\ldots,z_n:t_n |- P}
  {\Psi,x:s |- x(z_1\ldots z_n).P}
$\; (3)$
\end{pilisting}

dove le condizioni per le regole {\small res}, {\small out} e {\small inp}
sono rispettivamente

\vspace{3mm}
\indent
$
    (1) \; z \notin n(\Psi);
$
\vspace{3mm}
\\
\indent
 $
    (2) \; \lambda(s) = (t_1\ldots t_n),
           \Psi \simeq x : s,
           \Psi \simeq y_1 : t_1$ (per ogni $i$), $
           x = y_i$ implica $ \\
$
\indent
$
          \;\;\;\;\;
          s = t_i$ (per qualche $i$) e $
          y_i = y_j$ implica $t_i = t_j$ (per qualche $i$, $j$); $
$
\vspace{3mm}
\\
\indent
$
    (3) \; \lambda(s) = (t_1\ldots t_n),
           \Psi \simeq x : s$ e $z_1,\ldots,z_n \notin n(\Psi).
$
\vspace{3mm}

Si dice che un processo $P$ \emph{rispetta} un sorting $\lambda$ se esiste
un qualche ambiente $\Psi$ tale che \`e possibile usare le regole di
inferenza appena descritte per derivare la conclusione $\Psi |- P$.
