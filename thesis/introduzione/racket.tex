\subsection{Racket}

\emph{Racket} \`e un linguaggio di programmazione appartenente alla
famiglia Lisp.

Nato come implementazione del linguaggio di programmazione funzionale Scheme
sotto il nome PLT Scheme, si \`e successivamente discostato da questo
linguaggio abbastanza da convincere i suoi autori a modificarne il nome;
nonostante questo, Racket si allontana ben poco da Scheme, e pu\`o
ragionevolmente essere considerato una sua implementazione.

Le ragioni che hanno portato ad utilizzare Racket anzich\'e una delle
numerose implementazioni del linguaggio Scheme cos\`i come definito nello
standard R$^{5}$RS (Revised$^{5}$ Report on the Algorithmic Language
Scheme\cite{r5rs}, la versione pi\`u diffusamente supportata) sono
molteplici: innanzitutto, Scheme \`e un linguaggio progettato secondo
criteri di minimalismo, e se questo ha da una parte portato ad
un'encomiabile purezza del risultato, dall'altra rende necessario
l'utilizzo di librerie esterne per svolgere compiti quali la generazione di
parser, il pattern matching o lo sviluppo di una test suite; Racket, dal
canto suo, fornisce implementazioni complete e solide di tutte queste
funzionalit\`a.

Racket \`e estremamente ben documentato, e fornisce al programmatore un
sistema di typesetting, denominato \emph{Scribble}, che consente di
realizzare con facilit\`a documentazione della stessa qualit\`a.

Racket \`e \emph{Software Libero} rilasciato sotto licenza GNU LGPL, \`e
disponibile su tutte le principali piattaforme, ed \`e in grado di compilare
il codice sia in bytecode altamente efficiente che in formato eseguibile
nativo, oltre ad interpretarlo. Tutte queste caratteristiche rendono Racket
una piattaforma di sviluppo decisamente attraente.

La sintassi di Racket \`e fortemente influenzata dalla sintassi del
linguaggio Lisp, ed \`e basata su \emph{S-expressions}, ovvero espressioni
racchiuse tra parentesi tonde dove il primo elemento \`e la procedura
da chiamare e gli elementi successivi, se presenti, sono i suoi argomenti

\begin{lstlisting}
      (define greet "Hello!")
      (display greet)
      (newline)
\end{lstlisting}

Diversamente da quanto avviene con la maggior parte dei linguaggi di
programmazione, praticamente ogni parte di Racket, compresi gli operatori
matematici di base, \`e implementata sotto forma di funzione.

Questo fatto, unito alla peculiare sintassi di Racket, pu\`o inizialmente
causare confusione: \`e frequente incontrare espressioni simili a

\begin{lstlisting}
      (- p (* p p))
\end{lstlisting}

indubbiamente meno immediate di quanto la stessa espressione non sarebbe
se espressa in un linguaggio derivato da C.

Superato l'impatto iniziale, ci si rende conto che la sintassi estremamente
regolare di Racket fornisce un vantaggio durante l'analisi del codice, e
che l'avere la quasi totalit\`a del linguaggio implementato in termini di
funzioni consente una flessibilit\`a difficilmente raggiungibile in altri
linguaggi.

Si prenda ad esempio il seguente frammento di codice:

\begin{lstlisting}
      (define lst (list 1 2 3))
      (foldl + 0 lst) ; ==> 6
\end{lstlisting}

nella riga 1, la funzione \lstinline{list} viene chiamata con argomenti
\lstinline{1}, \lstinline{2} e \lstinline{3} per creare una lista, alla
quale viene assegnato il nome \lstinline{lst}.

Nella riga 2, la lista \lstinline{lst} viene manipolata dalla funzione
\lstinline{foldl}: questa funzione accetta come argomenti, nell'ordine,
una funzione a due argomenti, un valore con cui inizializzare
l'accumulatore, ed una lista su cui lavorare.

Il fatto che l'operatore matematico \lstinline{+} sia implementato come
funzione ci consente di passarlo direttamente alla \lstinline{foldl}, che
sommer\`a uno alla volta i numeri contenuti nella lista all'accumulatore,
ottenendo alla fine il valore \lstinline{6}. Questo esempio mostra anche
come in Racket le funzioni siano un tipo di dato di prim'ordine.

\`E possibile definire nuove funzioni utilizzando la sintassi
\lstinline{lambda}, che crea funzioni anonime, combinata con la sintassi
\lstinline{define}, oppure con una sintassi abbreviata. I seguenti esempi
sono totalmente equivalenti tra di loro:

\begin{lstlisting}
      (define (double x)
        (* 2 x))
\end{lstlisting}

\begin{lstlisting}
      (define double
        (lambda (x)
          (* 2 x)))
\end{lstlisting}

In quanto discendente del linguaggio Lisp (il cui nome sta per LISt
Processor), Racket pone una particolare enfasi sulla manipolazione di
liste, fornendo diverse primitive utili: le funzioni \lstinline{car} e
\lstinline{cdr} restituiscono rispettivamente il primo elemento di una
lista, e gli elementi rimanenti:

\begin{lstlisting}
      (define three (lambda () 3))
      (define lst (list 1 "two" three))
      (car lst) ; ==> 1
      (cdr lst) ; ==> (list "two" three)
\end{lstlisting}

Come si pu\`o vedere, una lista \`e in grado di contenere elementi
eterogenei: nel nostro caso, il numero \lstinline{1}, la stringa
\lstinline{"two"}, e una procedura che accetta zero argomenti e restituisce
il numero \lstinline{3}.

Altra procedura fondamentale di manipolazione liste \`e la \lstinline{map},
che applica una funzione ad ogni elemento di una lista, restituendo la
lista contenente gli elementi trasformati:

\begin{lstlisting}
      (define lst (list 6 2 4))
      (map (curry max 4) lst) ; ==> (list 6 4 4)
\end{lstlisting}

La chiamata a \lstinline{map} applica la funzione \lstinline{(curry max 4)}
ad ogni elemento di \lstinline{lst}, accumulando i risultati in una lista.

La funzione \lstinline{curry}, non definita nello standard R$^{5}$RS ma
fornita da Racket, permette di generare funzioni curryficate: lo stesso
effetto si sarebbe potuto ottenere, in maniera meno tersa, utilizzando la
sintassi \lstinline{lambda} nel seguente modo:

\begin{lstlisting}
      (define lst (list 6 2 4))
      (map (lambda (x) (max 4 x)) lst) ; ==> (list 6 4 4)
\end{lstlisting}

Questi esempi dovrebbero aver dato un'idea piuttosto precisa di come Racket
funzioni, e su quali concetti si basi.

Nelle pagine successive verr\`a analizzato il codice dell'implementazione
del type system per il $\pi$-calcolo, e si incontreranno molte altre
caratteristiche di questo linguaggio.
