\section{Racket}

\emph{Racket} \`e un linguaggio di programmazione appartenente alla
famiglia LISP.

Nato come implementazione del linguaggio di programmazione funzionale Scheme
sotto il nome PLT Scheme, si \`e successivamente discostato da questo
linguaggio abbastanza da convincere i suoi autori a modificarne il nome;
nonostante questo, Racket si allontana ben poco da Scheme, e pu\`o
ragionevolmente essere considerato una sua implementazione.

Le ragioni che hanno portato ad utilizzare Racket anzich\'e una delle
numerose implementazioni del linguaggio Scheme cos\`i come definito nello
standard R5RS sono molteplici: innanzitutto, Scheme \`e un linguaggio
progettato secondo criteri di minimalismo, e se questo ha da una parte
portato ad un'encomiabile purezza del risultato, dall'altra rende necessario
l'utilizzo di librerie esterne per svolgere compiti quali la generazione di
parser, il pattern matching o lo sviluppo di una test suite; Racket, dal
canto suo, fornisce implementazioni complete e solide di tutte queste
funzionalit\`a.

Racket \`e estremamente ben documentato, e fornisce al programmatore un
sistema di typesetting, denominato \emph{Scribble}, che consente di
realizzare con facilit\`a documentazione della stessa qualit\`a.

Racket \`e \emph{Software Libero} rilasciato sotto licenza GNU LGPL, \`e
disponibile su tutte le principali piattaforme, ed \`e in grado di compilare
il codice sia in bytecode altamente efficiente che in formato eseguibile
nativo, oltre ad interpretarlo. Tutte queste caratteristiche rendono Racket
una piattaforma di sviluppo decisamente attrattiva.

La sintassi di Racket \`e fortemente influenzata dalla sintassi del
linguaggio LISP, ed \`e basata su \emph{S-expressions}, ovvero espressioni
racchiuse tra parentesi tonde

\begin{lstlisting}
(define foo 42)
(display foo)
\end{lstlisting}

Diversamente da quanto avviene con la maggior parte dei linguaggi di
programmazione, praticamente ogni parte di Racket, compresi gli operatori
matematici di base, \`e implementata sotto forma di funzione.
