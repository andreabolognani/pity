\documentclass[a4paper]{article}

\usepackage[italian]{babel}
\usepackage{listings}
\usepackage{hyperref}
\usepackage{semantic}

% PDF file metadata
\hypersetup{pdftitle={Pity -- Pi-Calculus Type Inference},
            pdfauthor={Andrea Bolognani}}

% Pi-Calculus listing environment
\newenvironment{pilisting}
	{\vspace{3mm}\indent}
	{\vspace{3mm}}

% Environment for terminal interaction (use to wrap lstlisting)
\newenvironment{termlisting}
    {\lstset{language=C}}
    {\lstset{language=Racket}}

% Simple (and incomplete) Racket language definition
\lstdefinelanguage{Racket}
	{morekeywords={lambda,
                   define, struct, let, let*,
                   values,
                   equal?, not, and, or,
                   when, if, unless, cond,
                   list, car, cdr, map, foldl,
                   +, -, *, /,
                   string?,
                   display, newline, format, error},
     sensitive=true,
     morecomment=[l]{;},
     morestring=[b]{"},}

% Source code listings appearance
\lstset{language=Racket,
        basicstyle=\ttfamily\footnotesize,
        keywordstyle=\bf,
        emphstyle=\underbar,
        showstringspaces=false,
        numbers=none,
        numberstyle=\tiny,
        numbersep=-12pt,
        aboveskip=3mm,
        belowskip=3mm,
        lineskip=2pt,
        gobble=1}

% List of bindings we are defining
\lstset{emph={}}


\begin{document}

\title{Pity -- Pi-Calculus Type Checking}
\author{Andrea Bolognani}
\date{Ottobre 2010}

\maketitle


\begin{abstract}
Contenuto della tesi, spiegato molto brevemente e molto ad alto livello.
\end{abstract}


\clearpage

\tableofcontents

\clearpage


\section{Introduzione}

Introduzione al lavoro svolto. Riassume tutto il lavoro svolto, quindi
va scritta come ultima cosa.


\subsection{Il \texorpdfstring{$\pi$}{pi}-calcolo}

Il \emph{$\pi$-calcolo} \`e un formalismo matematico che consente di
modellare un insieme di processi paralleli e di studiare il loro
comportamento.

Si basa su un ristretto insieme di potenti astrazioni: alla base del
$\pi$-calcolo si trova il concetto di \emph{nome}, che rappresenta un
canale di comunicazione tra due processi. Nuovi canali di comunicazione
possono essere aperti tra due processi semplicemente inviando un nuovo
nome da un processo all'altro.

Le \emph{azioni} possibili sono

\begin{pilisting}
$
    \pi \;\; ::= \;\; \overline{x}\langle\tilde{y}\rangle
        \;\; |   \;\; x(\tilde{y})
$
\end{pilisting}

rispettivamente di output ed input, dove $\tilde{y}$ \`e in entrambi i casi
una lista non vuota $ y_1 \ldots y_n$ di nomi distinti tra loro.

I \emph{processi} vengono ottenuti combinando tra loro pi\`u azioni in uno
dei seguenti modi

\begin{pilisting}
$
    P \;\; ::= \;\; 0
      \;\; |   \;\; \pi.P
      \;\; |   \;\; P|P^{'}
      \;\; |   \;\; \nu z \, P
      \;\; |   \;\; !P
$
\end{pilisting}

Il processo nullo $0$ \`e utilizzato come punto di partenza per la
costruzione di qualsiasi processo di $\pi$-calcolo; ad ogni processo pu\`o
essere aggiunto come \emph{prefisso} un'azione, ottenendo come risultato un
processo che compie l'azione per poi comportarsi come il processo originale.

Un esempio \`e il processo

\begin{pilisting}
$
    \overline{a}\langle b\rangle.0
$
\end{pilisting}

che trasmette il nome $b$ sul canale $a$ prima di comportarsi come $0$,
ovvero fermarsi.

Tramite l'operatore $|$ \`e possibile creare una composizione di processi:
nel processo

\begin{pilisting}
$
    \overline{a}\langle b,c\rangle.0 \; | \;
    a(u,v).0
$
\end{pilisting}

il lato sinistro e quello destro dell'espressione sono entit\`a che agiscono
autonomamente.

In questo caso, il lato sinistro vuole inviare i nomi $b$ e $c$ tramite il
canale $a$, lo stesso che il lato destro usa per ricevere due nomi da
sostituire ad $u$ e $v$; una volta avvenuta la comunicazione, entrambi i
processi si riducono a $0$.

\`E evidente gi\`a da questo semplice esempio come, se si vuole che una
comunicazione abbia successo, sia necessario che entrambe le parti
concordino quantomeno sul numero di nomi da inviare: nel processo

\begin{pilisting}
$
    \overline{a}\langle x\rangle.0 \; | \;
    a(u,v).0
$
\end{pilisting}

il processo mittente invia un solo nome, mentre il processo destinatario
si aspetta di riceverne due. Il type system implementato ha proprio lo
scopo di garantire che una simile situazione non si venga mai a verificare.

L'operatore di \emph{restrizione} $\nu x$ consente di creare un nuovo nome
$x$, che pu\`o successivamente essere utilizzato per comunicare tra un
gruppo ristretto di processi. Nel sistema

\begin{pilisting}
$
    \nu x \, (
      \overline{x}\langle y\rangle.0 \; | \;
      x(z).0
    ) \; | \;
    a(b).0
$
\end{pilisting}

il primo ed il secondo processo sono in grado di comunicare tra di loro
utilizzando il nome $x$, cosa che non \`e possibile al terzo processo, che
si trova all'esterno dello scope di $x$ (rappresentato dalle parentesi che
racchiudono i primi due processi).

Nella modellazione di sistemi reali, raramente ci si trova ad analizzare
processi che, come quelli visti fino ad ora, sono in grado di eseguire una
singola comunicazione prima di ridursi al processo nullo: nella maggior
parte dei casi, il processo lavora in maniera ciclica.

Immaginando di voler descrivere, tramite il $\pi$-calcolo, il funzionamento
di un semplice Web server, si arriverebbe a qualcosa di simile a

\begin{pilisting}
$
    !s(c,q).(\nu r \, \overline{c}\langle r\rangle.0)
$
\end{pilisting}

dove $!$ \`e l'operatore di \emph{replicazione}, definito in maniera tale
che $!P$ si comporti come $P \; | \; !P$.

Il funzionamento \`e semplice: il server riceve sul canale $s$ i nomi $c$ e
$q$, che rappresentano rispettivamente il canale da usare per comunicare la
risposta al client e la sua richiesta. Una volta ricevuti questi dati, il
server procede a generare una risposta $r$, che invia al client sul nome $c$
precedentemente ricevuto, e a mettersi in attesa di una nuova richiesta.

Le occorrenze di $x$ in $\nu x \, P$ e in $ a(x).P$ sono \emph{leganti}:
questo significa che tutte le occorrenze successive di $x$ in $P$ fanno
riferimento allo stesso nome $x$. L'insieme dei nomi legati in un processo
$P$ si indica con $bn(P)$.

Le occorrenze di un nome che non sono nello scope di un'occorrenza legante
sono \emph{libere}: ad esempio, l'occorrenza di $a$ in
$\overline{a}\langle b\rangle.P$ \`e libera, mentre in
$\nu a \; \overline{a}\langle b\rangle.P$ il nome $a$ non ha occorrenze
libere. L'insieme dei nomi liberi in $P$ si indica con $fn(P)$.

L'insieme dei nomi di $P$ si definisce come $n(P) = fn(P) \cup bn(P)$.


\section{Racket}

Come si pu\`o capire gi\`a dal primo impatto con $\pi$-calcolo, la sua
definizione formale \`e naturalmente ricorsiva: si parte infatti dal
processo nullo $0$, al quale si pu\`o aggiungere un'azione per ottenere
un nuovo processo, nel quale si pu\`o restringere un nome per ottenere un
nuovo processo, e cos\`i via.

Le regole di inferenza che costituiscono il type system sono a loro volta
basate sulla ricorsione: ad esempio, per verificare che il processo
$P \; | \; Q$ rispetti un sorting, \`e necessario prima verificare che
$P$ e $Q$ rispettino singolarmente tale sorting.

Quando ci si trova a dover descrivere queste interazioni in un programma,
viene quindi naturale utilizzare un linguaggio di programmazione
funzionale, dove la ricorsione \`e il principale metodo per effettuare
computazione.

Tra i vari linguaggi di programmazione funzionale esistenti, la scelta
\`e ricaduta su \emph{Racket}, un linguaggio di programmazione
appartenente alla famiglia Lisp.

Racket \`e un dialetto di \emph{Scheme}, una delle tre principali
varianti del linguaggio Lisp; le restanti due sono Common Lisp e Emacs
Lisp.

Emacs Lisp costituisce un caso a s\'e, in quanto \`e integrato con
l'editor di testo Emacs al punto tale che non \`e possibile
utilizzarlo al di fuori di esso; questo fatto lo rende ovviamente poco
appetibile come linguaggio di programmazione al di l\`a dell'ambito
delle estensioni all'editor.

Common Lisp non ha questo tipo di problemi, ma soffre di alcune
limitazioni e particolarit\`a dovute dall'et\`a del suo design, come
ad esempio il fatto di utilizzare due spazi di indirizzamento diversi
per variabili e funzioni, e di gestire lo scoping di una variabile in
maniera diversa a seconda di come viene dichiarata.

Queste caratteristiche, unite all'enorme numero di moduli (talvolta
scarsamente documentati) di cui Common Lisp dispone, rendono l'approccio
a questo linguaggio poco amichevole.

Il linguaggio Scheme, invece, ha un design minimalista, direttamente
derivato dal $\lambda$-calcolo, che parte da un numero molto ridotto di
costrutti base dal quale poi viene derivato tutto il linguaggio.

Le variabili e le funzioni condividono un singolo spazio di
indirizzamento, e tutte le variabili hanno scoping lessicale; la sintassi
\`e estremamente regolare e favorisce parole inglesi esistenti piuttosto
che usare abbreviazioni o acronimi.

Tutte queste caratteristiche rendono possibile imparare Scheme in pochi
giorni, e acquisire dimestichezza con praticamente tutti i costrutti che
lo compongono in un paio di settimane. Per questo motivo, Scheme \`e
frequentemente utilizzato nell'ambito dell'educazione.

Il linguaggio Scheme \`e definito formalmente da una serie di report: il
pi\`u recente \`e R$^{6}$RS (Revised$^{6}$ Report on the Algorithmic
Language Scheme\cite{r6rs}), pubblicato nel 2009, mentre la versione che
\`e pi\`u ampiamente supportata \`e la precedente R$^{5}$RS (Revised$^{5}$
Report on the Algorithmic Language Scheme\cite{r5rs}), del 1998.

Al di fuori delle specifiche ufficiali, esiste un processo chiamato
SRFI (Scheme Requests For Implementation), che consente ai membri della
comunit\`a di utilizzatori di proporre estensioni al linguaggio, complete
di implementazione, che possono essere incluse nello standard successivo
oppure distribuite come estensioni autonome.

Molte delle implementazioni di Scheme disponibili supportano lo
standard R$^{5}$RS pi\`u un certo numero di SRFI. Alcune di esse sono
in grado di funzionare utilizzando una quantit\`a estremamente ridotta
di memoria, il che, unito alla sintassi minimale del linguaggio, le
rende particolarmente adatte ad essere incluse in un programma pi\`u
ampio in qualit\`a di linguaggio di estensione.

Racket \`e nato come implementazione del linguaggio Scheme sotto il nome
PLT Scheme, si \`e successivamente discostato da questo linguaggio
abbastanza da convincere i suoi autori a modificarne il nome; nonostante
questo, Racket non presenta estensioni incompatibili con R$^{5}$RS, e
se ne differenzia principalmente offrendo estensioni e librerie
aggiuntive.

Proprio l'ampia disponibilit\`a di librerie ben documentate \`e stato
uno dei fattori chiave nella scelta di Racket come linguaggio da
utilizzare per l'implementazione: oltre a supportare un ampio numero
di SRFI, Racket comprende un sistema di moduli che consente di separare
il proprio codice in unit\`a funzionali indipendenti, un sistema di
contratti che garantisce la corretta tipizzazione del codice, un
generatore di lexer e parser, una completa libreria di pattern matching
e una libreria per la scrittura ed esecuzione di test suite.

La documentazione di Racket, come gi\`a accennato, \`e estremamente
completa e ben organizzata; inoltre, \`e possibile per il programmatore
utilizzare \emph{Scribble}, il sistema di typesetting utilizzato per
generare tale documentazione, con il proprio codice.

Racket \`e \emph{Software Libero} rilasciato sotto licenza GNU LGPL, ed
\`e disponibile su tutte le principali piattaforme.

Racket fornisce tre modalit\`a di esecuzione di un programma:
interpretazione, compilazione in bytecode, oppure generazione di un
eseguibile nativo che permette di far girare un'applicazione scritta in
Racket senza richiedere l'installazione dell'ambiente di sviluppo.

La sintassi di Racket \`e fortemente influenzata dalla sintassi del
linguaggio Lisp, ed \`e basata su \emph{S-expressions}, ovvero espressioni
racchiuse tra parentesi tonde dove il primo elemento \`e la procedura
da chiamare e gli elementi successivi, se presenti, sono i suoi
argomenti.

Il classico programma di Hello World si scrive in Racket come

\begin{lstlisting}
    (define greet "Hello, World!")
    (display greet)
    (newline)
\end{lstlisting}

Questo semplice frammento di codice definisce una nuova variabile
chiamata \lstinline{greet}, alla quale viene assegnata la stringa
\lstinline{"Hello!"}; la stringa viene visualizzata utilizzando la
procedura \lstinline{display}, e ad essa viene fatto seguire un
ritorno a capo tramite la procedura \lstinline{newline}.

La sintassi prefissa di Racket pu\`o inizialmente causare confusione,
in particolare quando si ha a che fare con espressioni matematiche,
come

\begin{lstlisting}
    (- p (* p p))
\end{lstlisting}

che appare meno immediata di quanto la stessa espressione non sarebbe
se espressa in un linguaggio derivato da C; con il tempo, ci si abitua
a questo tipo di sintassi, che risulta comunque molto meno limitante
al di fuori dell'ambito della pura algebra.

In alcuni casi, l'utilizzo della sintassi prefissa si rivela essere un
vantaggio: ad esempio, sfruttando il fatto che l'operatore \lstinline{+}
accetta un numero variabile di argomenti, si possono sommare sette numeri
in maniera molto concisa:

\begin{lstlisting}
    (+ 1 8 12 6 5 9 3)
\end{lstlisting}

In ogni caso, una volta superato l'impatto iniziale, la sintassi di
Racket non causa alcun problema di comprensione, soprattutto se il
codice \`e adeguatamente indentato.

Diversamente da quanto avviene con la maggior parte dei linguaggi di
programmazione, praticamente ogni parte di Racket, compresi gli operatori
matematici di base, \`e implementata sotto forma di funzione.

Si prenda ad esempio il seguente frammento di codice:

\begin{lstlisting}
    (define lst (list 1 2 3))
    (foldl + 0 lst) ; ==> 6
\end{lstlisting}

nella prima riga, la procedura standard \lstinline{list} viene chiamata
con argomenti \lstinline{1}, \lstinline{2} e \lstinline{3} per creare una
lista, che viene assegnata alla variabile \lstinline{lst}.

Nella riga successiva, la lista \lstinline{lst} viene manipolata dalla
procedura \lstinline{foldl}: questa procedura accetta come argomenti,
nell'ordine, una funzione a due argomenti, un valore con cui
inizializzare l'accumulatore, ed una lista su cui lavorare.

Il fatto che l'operatore matematico \lstinline{+} sia implementato come
funzione ci consente di passarlo direttamente alla \lstinline{foldl}, che
sommer\`a uno alla volta i numeri contenuti nella lista all'accumulatore,
ottenendo alla fine il valore \lstinline{6}. Questo esempio mostra anche
come in Racket le funzioni siano un tipo di dato di prim'ordine.

\`E possibile definire nuove funzioni utilizzando la sintassi
\lstinline{lambda}, che crea funzioni anonime, combinata con la sintassi
\lstinline{define}, oppure con una sintassi abbreviata. I seguenti esempi
sono totalmente equivalenti tra di loro:

\begin{lstlisting}
    (define (double x)
      (* 2 x))
\end{lstlisting}

\begin{lstlisting}
    (define double
      (lambda (x)
        (* 2 x)))
\end{lstlisting}

Le procedure dichiarate possono contenere variabili libere: al momento
della chiamata, Racket cerca un binding nell'ambiente di definizione
della procedura, e utilizza quel valore:

\begin{lstlisting}
    (define (three) (+ two 1))
    (three) ; ==> ERRORE
    (define two 2)
    (three) ; ==> 3
\end{lstlisting}

Questa caratteristica pu\`o essere usata per costruire, ad esempio, una
funzione che mantiene stato tra una chiamata e l'altra, come un semplice
contatore:

\begin{lstlisting}
    (define next! #f)
    (let ([number 0])
      (set! next! (lambda ()
                    (set! number (+ 1 number))
                    number)))
    (next!) ; ==> 1
    (next!) ; ==> 2
    (define number 9)
    (next!) ; ==> 3
\end{lstlisting}

Inizialmente, si definisce una variabile \lstinline{next!}, inizializzata
con il valore \lstinline{#f}, che servir\`a per contenere la procedura
contatore.

Si usa poi la sintassi \lstinline{let} per creare un nuovo ambiente,
nel quale il valore \lstinline{0} viene assegnato alla variabile
\lstinline{number}. Nello scope di questo assegnamento, si assegna
alla variabile \lstinline{next!} (nello scope esterno, visto che non ci
sono dichiarazioni nello scope interno che la nascondono) una nuova
procedura, che incrementa il valore della variabile \lstinline{number}
prima di restituirlo.

Nello scope esterno, ogni chiamata alla funzione \lstinline{next!}
restituisce un nuovo valore; inoltre, dichiarare una variabile di nome
\lstinline{number} non influenza il valore restituito dalla
funzione \lstinline{next!}. Questo perch\'e la funzione \lstinline{next!}
fa sempre riferimento alla variabile \lstinline{number} dichiarata
nello scope interno.

La procedura \lstinline{set!} permette di modificare il valore di una
variabile. Il simbolo \lstinline{!} indica appunto una procedura non
\emph{pura}; \`e convenzione in Racket utilizzare questo simbolo nel
nome delle funzioni che alterano dello stato globale. Le funzioni non
pure vengono solitamente evitate in Racket, preferendo ad esse procedure
che operano in maniera non distruttiva e prive di effetti secondari.

Cos\`i come le procedure non pure per convenzione terminano con il
simbolo \lstinline{!}, in Racket viene utilizzata la convenzione di
terminare con il simbolo \lstinline{?} i predicati, ovvero le procedure
che restituiscono valori di verit\`a, come \lstinline{list?} o
\lstinline{procedure?}:

\begin{lstlisting}
    (number? "2") ; ==> #f
    (list? (list 1 2 3)) ; ==> #t
    (false? procedure?) ; ==> #f
\end{lstlisting}

Si pu\`o vedere nell'ultima riga che \lstinline{procedure?} \`e
considerato un valore di verit\`a positivo: in Racket ogni valore
diverso da \lstinline{#f} \`e considerato tale.

I valori di verit\`a possono essere utilizzati nei costrutti
condizionali:

\begin{lstlisting}
    (if (and (list? a)
             (not (null? a)))
        (display (car a))
        (display a))
\end{lstlisting}

Se la variabile \lstinline{a} contiene una lista non vuota, viene
mostrato il primo elemento della lista, in caso contrario viene
mostrato il contenuto della variabile.

Contrariamente a quanto succede in altri linguaggi, in Racket bisogna
sempre specificare entrambi i rami della sintassi \lstinline{if}.
Questo causa problemi se si vogliono avere pi\`u chiamate a funzione
nel corpo di uno dei due rami, ma si pu\`o facilmente aggirare questa
limitazione usando la sintassi \lstinline{begin}:

\begin{lstlisting}
    (define (display-list lst)
      (if (null? lst)
          (newline)
          (begin
            (display (car lst))
            (display " ")
            (display-list (cdr lst)))))
\end{lstlisting}

Un'altra sintassi, pi\`u generica, che si pu\`o utilizzare per i
costrutti condizionali \`e \lstinline{cond}, che consente un numero
arbitrario di coppie condizione-risultato:

\begin{lstlisting}
    (cond
      [(number? a)        a]
      [(string->number a) (string->number a)]
      [else               0])
\end{lstlisting}

Le condizioni vengono valutate dalla prima all'ultima, e appena si
trova una condizione verificata viene restituito il valore
corrispondente; se \`e presente una clausola \lstinline{else} e
nessuna condizione \`e verificata, viene restituito il valore
assegnato alla clausola \lstinline{else}.

In questo caso, se la variabile \lstinline{a} contiene un numero, il
numero viene restituito; se invece \`e una stringa che pu\`o essere
convertita in un numero, viene restituito il valore ottenuto tramite
la conversione; se nessuna delle condizioni precedenti \`e verificata,
viene restituito il valore \lstinline{0}.

La procedura \lstinline{string->number} \`e una delle numerose procedure
di conversione tra tipi fornite da Racket. Convenzionalmente, queste
procedure hanno come nome il tipo di partenza, i simboli \lstinline{->}
e il tipo di arrivo. Quando la conversione fallisce (ad esempio se viene
richiesta la conversione della stringa \lstinline{"foo"} in un numero)
il valore restituito \`e solitamente \lstinline{#f}.

In quanto discendente del linguaggio Lisp (il cui nome sta per LISt
Processor), Racket pone una particolare enfasi sulla manipolazione di
liste, fornendo diverse primitive utili: le funzioni \lstinline{car} e
\lstinline{cdr} restituiscono rispettivamente il primo elemento di una
lista, e gli elementi rimanenti:

\begin{lstlisting}
    (define three (lambda () 3))
    (define lst (list 1 "two" three))
    (car lst) ; ==> 1
    (cdr lst) ; ==> (list "two" three)
\end{lstlisting}

Come si pu\`o vedere, una lista \`e in grado di contenere elementi
eterogenei: nel nostro caso, il numero \lstinline{1}, la stringa
\lstinline{"two"}, e una procedura che accetta zero argomenti e restituisce
il numero \lstinline{3}.

In realt\`a le primitive \lstinline{car} e \lstinline{cdr} operano non
sulle liste, ma sulle coppie, che sono i costituenti fondamentali delle
liste, e che si possono creare con la primitiva \lstinline{cons}:

\begin{lstlisting}
    (define v (cons 2 "3"))
    (car v) ; ==> 2
    (cdr v) ; ==> "3"
\end{lstlisting}

come si pu\`o vedere, \lstinline{car} e \lstinline{cdr} restituiscono
rispettivamente il primo e il secondo elemento di una coppia; le liste
vengono costruite partendo dalla lista vuota \lstinline{(list)}

\begin{lstlisting}
    (cons "3" (list)) ; ==> (list "3")
\end{lstlisting}

e ulteriori elementi possono essere aggiunti alla lista con altre chiamate
a \lstinline{cons}:

\begin{lstlisting}
    (cons 2 (cons "3" (list))) ; ==> (list 2 "3")
\end{lstlisting}

Esistono varianti combinate delle primitive \lstinline{car} e
\lstinline{cdr}, come ad esempio \lstinline{cadr}, che restituisce il
secondo elemento di una lista, o \lstinline{cddar}, che accetta una lista
il cui primo elemento \`e a sua volta una lista, e restituisce tutti i
valori contenuti nel primo elemento tranne i primi due:

\begin{lstlisting}
    (define lst (list (list 1 2 3) 4 5 6))
    (cadr lst) ; ==> 4
    (cddar lst) ; ==> (list 3)
\end{lstlisting}

\`E possibile selezionare solamente alcuni elementi di una lista tramite
la procedura \lstinline{filter}, che accetta un predicato ed una lista, e
restituisce tutti i valori della lista che soddisfano il predicato:

\begin{lstlisting}
    (define lst (list -1 5 0 -22 10 -4))
    (define (positive-even-number? x)
      (and (number? x)
           (positive? x)
           (even? x)))
    (filter positive-even-number? lst) ; ==> (list 10)
\end{lstlisting}

Altra procedura fondamentale di manipolazione liste \`e la \lstinline{map},
che applica una funzione ad ogni elemento di una lista, restituendo la
lista contenente gli elementi trasformati:

\begin{lstlisting}
    (define lst (list 6 2 4))
    (map (curry max 4) lst) ; ==> (list 6 4 4)
\end{lstlisting}

La chiamata a \lstinline{map} applica la funzione \lstinline{(curry max 4)}
ad ogni elemento di \lstinline{lst}, accumulando i risultati in una lista.

La funzione \lstinline{curry}, non definita nello standard R$^{5}$RS ma
fornita da Racket, permette di generare funzioni curryficate: lo stesso
effetto si sarebbe potuto ottenere, in maniera meno tersa, utilizzando la
sintassi \lstinline{lambda} nel seguente modo:

\begin{lstlisting}
    (define lst (list 6 2 4))
    (map (lambda (x) (max 4 x)) lst) ; ==> (list 6 4 4)
\end{lstlisting}

Si pu\`o utilizzare \lstinline{apply} per chiamare una procedura con
gli argomenti contenuti in una lista. Per calcolare la somma dei
numeri contenuti in una lista, al posto di usare \lstinline{foldl}
come visto sopra, si pu\`o usare

\begin{lstlisting}
    (define lst (list 1 2 3))
    (apply + lst) ; ==> 6
\end{lstlisting}

\lstinline{apply}, unita alla possibilit\`a di dichiarare una funzione
che riceve i suoi argomenti in una lista, consente di scrivere
facilmente funzioni che trasformano altre funzioni, come ad esempio la
seguente procedura \lstinline{flip}:

\begin{lstlisting}
    (define (flip proc)
      (lambda rest (apply proc (reverse rest))))
\end{lstlisting}

Con la sintassi \lstinline{(lambda rest ...)} si crea una funzione che
riceve i suoi argomenti nella lista \lstinline{rest}: gli argomenti
sono rovesciati con la chiamata a \lstinline{reverse}, e la procedura
originale \lstinline{proc} viene chiamata usando \lstinline{apply}.

Un esempio dell'utilizzo di \lstinline{flip}:

\begin{lstlisting}
    ((flip list) 1 2 3) ; ==> (list 3 2 1)
\end{lstlisting}

Come si \`e potuto vedere, in Racket \`e possibile eseguire potenti
manipolazioni su procedure con poche righe di codice.

Dal momento che ogni computazione viene eseguita tramite l'applicazione
di procedure, \`e importante che la chiamata a procedura sia
un'operazione che richiede poche risorse. Per assicurare che questo
avvenga, lo standard R$^{5}$RS richiede che ogni implementazione supporti
l'ottimizzazione della chiamata in coda, in modo che codice scritto
in maniera adeguata possa eseguire un numero illimitato di ricorsioni
utilizzando una quantit\`a fissa di memoria.

Ad esempio, l'implementazione pi\`u ovvia del fattoriale

\begin{lstlisting}
    (define (fact x)
      (if (zero? x)
          1
          (* x (fact (- x 1)))))
\end{lstlisting}

pu\`o essere riscritta in modo da utilizzare la chiamata in coda

\begin{lstlisting}
    (define (fact x)
      (letrec ([do (lambda (x acc)
                     (if (zero? x)
                         acc
                         (do (- x 1) (* x acc))))])
        (do x 1)))
\end{lstlisting}

Mentre la prima implementazione usa tanta pi\`u memoria quanto pi\`u \`e
grande il numero di cui viene calcolato il fattoriale, la seconda
implementazione \`e in grado di calcolare il fattoriale di qualsiasi
numero utilizzando la stessa quantit\`a di memoria, oltre che risultare
spesso pi\`u veloce.

Il pattern matching \`e disponibile in Racket tramite la sintassi
\lstinline{match}:

\begin{lstlisting}
    (match f
      [(list _ b ...) (apply * b)]
      [(cons a b)     (+ a b)]
      [else           0])
\end{lstlisting}

Questo particolare pattern restituisce il prodotto di tutti gli elementi
di \lstinline{f} tranne il primo, se \lstinline{f} \`e una lista; oppure
la somma dei due elementi che compongono \lstinline{f}, se \lstinline{f}
\`e una coppia; \lstinline{0} in caso contrario.

Il pattern matching funziona anche sulle strutture definite dall'utente:
se \`e stata dichiarata la struttura \lstinline{point} tramite

\begin{lstlisting}
    (struct point (x y))
\end{lstlisting}

si possono estrarre e manipolare comodamente le singole coordinate usando
la sintassi \lstinline{match} come in

\begin{lstlisting}
    (match (point 2 5)
      [(point a b) (point (+ b 1) (- a 1))]
      [else        (point 0 0)]) ; ==> (point 6 1)
\end{lstlisting}

\`E da notare che, contrariamente a quanto accade ad esempio con il
linguaggio OCaml, non viene effettuato alcun controllo per garantire che
un pattern sia esaustivo: ad esempio, il pattern

\begin{lstlisting}
    (match (list)
      [(list a b ...) (cons a (reverse b))])
\end{lstlisting}

genera un errore a runtime, dal momento che non \`e stata dichiarata una
clausola da applicare sulla lista vuota. Al programmatore \`e quindi
richiesta una maggiore cura ed attenzione nella definizione dei pattern.

Uno strumento che pu\`o aiutare ad alleviare questo problema \`e il
sistema di contratti fornito da Racket, che permette di garantire la
corretta tipizzazione dei parametri di una procedura.

Si prenda come esempio la seguente procedura:

\begin{lstlisting}
    (define (make-it-double str)
      (string-append str str))
\end{lstlisting}

Se questa procedura viene chiamata con un argomento che non \`e una
stringa, al momento di chiamare \lstinline{string-append} viene sollevato
un errore, ed il programma termina.

\`E possibile assegnare un contratto alla procedura in fase di
esportazione: usando la sintassi

\begin{lstlisting}
    (provide/contract
      [make-it-double (string? . -> . string?)])
\end{lstlisting}

ci si assicura che ogni chiamata alla procedura
\lstinline{make-it-double} all'esterno del modulo avvenga con parametri
del tipo atteso.

All'interno del modulo il contratto non viene controllato, il che
consente una maggiore flessibilit\`a e in generale non causa problemi,
visto che il codice all'interno di un modulo solitamente chiama le
procedure interne al modulo con i parametri corretti.

La flessibilit\`a del sistema di contratti di Racket consente, nel
caso lo si ritenesse necessario, di proteggere con un contratto anche
le chiamate effettuate dall'interno del modulo; basta utilizzare la
sintassi \lstinline{define/contract} nel seguente modo:

\begin{lstlisting}
    (define/contract (make-it-double str)
                     (string? . -> . string?) '()
      (string-append str str))
\end{lstlisting}

per garantire che ogni singola chiamata alla procedura venga fatta con
parametri di tipo compatibile, anche all'interno del modulo.

Tutte le procedure che formano la libreria standard di Racket sono
protette da un contratto, che viene visualizzato nella documentazione.


\section{Type system per il \texorpdfstring{$\pi$}{pi}-calcolo}

Regole di inferenza per il $\pi$-calcolo:

\vspace{3 mm}
\indent\indent
\inference[nil]
  {}
  {\Psi |- 0}

\vspace{3 mm}
\indent\indent
\inference[rep]
  {\Psi |- P}
  {\Psi |- !P}

\vspace{3 mm}
\indent\indent
\inference[par]
  {\Psi |- P_1 & \Psi |- P_2}
  {\Psi |- P_1|P_2}

\vspace{3 mm}
\indent\indent
\inference[res]
  {\Psi,z:s |- P}
  {\Psi |- (z)P}

\vspace{3 mm}
\indent\indent
\inference[out]
  {\Psi |- P}
  {\Psi,x:s,y_1:t_1,\ldots,y_n:t_n |- \overline{x}\langle y_1\ldots y_n\rangle.P}

\vspace{3 mm}
\indent\indent
\inference[inp]
  {\Psi,z_1:t_1,\ldots,z_n:t_n |- P}
  {\Psi,x:s |- x(z_1\ldots z_n).P}

\vspace{3 mm}
Alcune di queste regole hanno delle condizioni.



\section{Conclusioni}

Simile all'introduzione, riassume il lavoro svolto e spiega in che
modo si potrebbe andare avanti a sviluppare il lavoro.

\end{document}
