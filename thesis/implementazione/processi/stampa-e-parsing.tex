\subsubsection{Stampa e parsing}

Avendo dichiarato tutti i tipi necessari, \`e ora possibile rappresentare
ogni processo di $\pi$-calcolo si desideri.

Ad esempio, una rappresentazione del processo

\begin{pilisting}
$
    ! \; \nu z \; (
      \overline{z}\langle q\rangle.0 \; | \;
      z(w).0
    )
$
\end{pilisting}

pu\`o essere ottenuta valutando l'espressione

\begin{lstlisting}
    (replication
      (restriction (name "z")
                   (composition (prefix
                                  (output (name "z")
                                          (list (name "q")))
                                  (nil))
                                (prefix
                                  (input (name "z")
                                         (list (name "w")))
                                  (nil)))))
\end{lstlisting}

Questa rappresentazione \`e senza dubbio fedele al processo originale;
tuttavia, l'eleganza e la chiarezza dell'espressione matematica sono
andate perdute.

Per evitare che la dichiarazione di processi diventi un processo lungo
e tedioso, verr\`a utilizzata una semplice rappresentazione testuale,
molto vicina a quella matematica, che consentir\`a di scrivere e leggere
velocemente processi: utilizzando questa rappresentazione, il processo
appena visto pu\`o essere scritto come

\begin{lstlisting}
    !(z)(z<q>.0|z(w).0)
\end{lstlisting}

Verr\`a definita come prima cosa la conversione da processo a stringa,
e successivamente quella opposta, da stringa a processo.

La procedura che converte un processo in una stringa \`e
\lstinline{process->string}, cos\`i definita:

\begin{lstlisting}
    ; process->string : process? -> string?
    (define (process->string process)
      (match process
        [(nil)             (process->string/nil)]
        [(input x y)       (process->string/input x y)]
        [(output x y)      (process->string/output x y)]
        [(prefix a p)      (process->string/prefix a p)]
        [(restriction x p) (process->string/restriction x p)]
        [(replication p)   (process->string/replication p)]
        [(composition p q) (process->string/composition p q)]))
\end{lstlisting}

che delega la conversione a delle procedure specializzate

\begin{lstlisting}
    ; Rappresentazione testuale del processo nullo
    (define (process->string/nil)
      "0")

    ; Rappresentazione testuale di un'azione
    (define (process->string/action a)
      (match a
        [(input x y)  (process->string/input x y)]
        [(output x y) (process->string/output x y)]))

    ; Rappresentazione testuale di un'azione di input
    (define (process->string/input x y)
      (string-append (name->string x) "("
                     (name-list->string y) ")"))

    ; Rappresentazione testuale di un'azione di output
    (define (process->string/output x y)
      (string-append (name->string x) "<"
                     (name-list->string y) ">"))

    ; Rappresentazione testuale di un prefisso
    (define (process->string/prefix a p)
      (string-append (process->string a)
                     "."
                     (if (composition? p)
                         (enclose (process->string p))
                         (process->string p))))

    ; Rappresentazione testuale di una restrizione
    (define (process->string/restriction x p)
      (string-append (enclose (name->string x))
                     (if (composition? p)
                         (enclose (process->string p))
                         (process->string p))))

    ; Rappresentazione testuale di una replicazione
    (define (process->string/replication p)
      (string-append "!"
                     (if (composition? p)
                         (enclose (process->string p))
                         (process->string p))))

    ; Rappresentazione testuale di una composizione
    (define (process->string/composition p q)
      (string-append (process->string p)
                     "|"
                     (process->string q)))
\end{lstlisting}

Le procedure di conversione sono molto semplici; l'unico fattore degno
di nota \`e l'utilizzo della procedura privata \lstinline{enclose} quando
si converte un prefisso, una restrizione o una replicazione: tale
procedura, definita come

\begin{lstlisting}
    (define (enclose stuff)
      (string-append "(" stuff ")"))
\end{lstlisting}

garantisce che la precedenza degli operatori sia salvaguardata.

Se le composizioni non venissero adeguatamente racchiuse tra parentesi,
il processo usato come esempio verrebbe rappresentato come

\begin{lstlisting}
    !(z)z<q>.0|z(w).0
\end{lstlisting}

che ha un significato ben diverso da quello originale (il nome $z$ \`e
ristretto solamente nel lato sinistro della composizione; allo stesso
modo, solo il lato sinistro \`e oggetto di replicazione).

Per effettuare la conversione da stringa a processo si crea un parser
utilizzando il modulo \lstinline{parser-tools} di Racket.

Come prima cosa \`e necessario creare un \emph{lexer}: il modulo
privato \lstinline{common-lexer} contiene l'implementazione.

Vengono definite una serie di abbreviazioni per semplificare la
definizione dei pattern utilizzati per estrarre le varie parti della
stringa sorgente:

\begin{lstlisting}
    (define-lex-abbrevs
      (letter (union (char-range #\a #\z)
                     (char-range #\A #\Z)))
      (digit  (char-range #\0 #\9))
      (space  (union #\tab
                     #\space))
      (id     (concatenation (repetition 1 +inf.0 letter)
                             (repetition 0 +inf.0 digit))))
\end{lstlisting}

L'abbreviazione \lstinline{letter} corrisponde ad un carattere alfabetico
maiuscolo o minuscolo; \lstinline{digit} ad una cifra; \lstinline{space}
a uno spazio o ad un carattere di tabulazione; \lstinline{id} ad una
stringa costituita da uno o pi\`u caratteri alfabetici seguiti da zero o
pi\`u cifre, ovvero ad una stringa che soddisfa il predicato
\lstinline{id-string?}.

Vengono poi definiti una serie di \emph{token}, la quasi totalit\`a dei
quali privi di contenuto:

\begin{lstlisting}
    (define-empty-tokens common-symbols (EOF NIL DOT COMMA PIPE BANG
                                         COLON SEMICOLON EQ
                                         LP RP LAB RAB LCB RCB))
    (define-tokens       common-values  (ID))
\end{lstlisting}

Infine viene dichiarata la procedura di lexing vera e propria,
assegnando ad ogni token il pattern che lo contraddistingue:

\begin{lstlisting}
    (define common-lexer
      (lexer
        [id     (token-ID lexeme)]
        ["0"    (token-NIL)]
        ["."    (token-DOT)]
        [","    (token-COMMA)]
        ["|"    (token-PIPE)]
        ["!"    (token-BANG)]
        [":"    (token-COLON)]
        [";"    (token-SEMICOLON)]
        ["="    (token-EQ)]
        ["("    (token-LP)]
        [")"    (token-RP)]
        ["<"    (token-LAB)]
        [">"    (token-RAB)]
        ["{"    (token-LCB)]
        ["}"    (token-RCB)]
        [space  (common-lexer input-port)] ; Ignora gli spazi
        [(eof)  (token-EOF)]))
\end{lstlisting}

Si pu\`o notare che i simboli riconosciuti dal lexer sono pi\`u di
quelli necessari a definire la rappresentazione testuale dei processi:
questo perch\'e lo stesso lexer verr\`a successivamente utilizzato per
riconoscere le rappresentazioni testuali di altre entit\`a.

Il parser \`e definito da

\begin{lstlisting}
    (define process-parser
      (parser

        (start  process)
        (end    EOF)
        (tokens common-symbols common-values)
        (precs  (right PIPE) (nonassoc BANG))
        (error  (lambda (a b c) (void)))

        (grammar

          (process
            [(agent)                $1]
            [(BANG process)         (replication $2)]
            [(process PIPE process) (composition $1 $3)])

          (agent
            [(NIL)                  (nil)]
            [(action DOT agent)     (prefix $1 $3)]
            [(LP name RP agent)     (restriction $2 $4)]
            [(LP process RP)        $2])

          (action
            [(name LP names RP)     (input $1 $3)]
            [(name LAB names RAB)   (output $1 $3)])

          (names
            [(name)                 (list $1)]
            [(name COMMA names)     (list* $1 $3)])

          (name
            [(ID)                   (name $1)]))))
\end{lstlisting}

Le precedenze degli operatori di composizione e replicazione sono
assegnate esplicitamente usando la sintassi \lstinline{precs}, mentre
la precedenza dell'operatore di restrizione \`e data implicitamente
all'interno della grammatica.

La procedura pubblica \lstinline{string->process} accetta una stringa
e utilizza il parser appena definito per estrarne un processo:

\begin{lstlisting}
    ; string->process : string? -> process?
    (define (string->process str)
      (with-handlers ([exn:fail?
                       (lambda (e)
                         (raise (exn:fail
                                  "Error while parsing process"
                                  (exn-continuation-marks e))))])
        (let ([ip (open-input-string str)])
          (process-parser (lambda () (common-lexer ip))))))
\end{lstlisting}

Dalla stringa input \lstinline{str} viene ottenuta una \emph{porta}
tramite la chiamata a \lstinline{open-input-string}: la porta \`e il
tipo utilizzato da Racket per rappresentare una variabile dalla sulla
quale si possono effettuare operazioni di input e output, in maniera
analoga a quanto avviene con i file descriptor nel linguaggio C.

Visto che il parsing pu\`o fallire, ad esempio nel caso in cui la
stringa passata come input contenga caratteri non accettati dal lexer,
la chiamata al parser viene protetta dalla sintassi
\lstinline{with-handlers}, che permette di gestire le eccezioni
sollevate dalle funzioni chiamate al suo interno. In questo caso,
le eventuali eccezioni sono gestite semplicemente sollevando una
nuova eccezione generica.

Le procedure \lstinline{process->string} e \lstinline{string->process}
sono scritte in modo che, dato un processo \lstinline{p}, il confronto
tramite \lstinline{equal?} del processo restituito dalla chiamata
\lstinline{(string->process (process->string p))} e \lstinline{p} dia
esito positivo.

La possibilit\`a di utilizzare la rappresentazione testuale definita
da queste due procedure come interfaccia per l'utente rende
l'interazione con il sistema molto pi\`u naturale.
