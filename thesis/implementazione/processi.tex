\subsection{Processi}

Il modulo \lstinline{pity/process} contiene l'implementazione dei processi,
che costituisce la parte pi\`u importante, oltre che pi\`u complessa, della
libreria sviluppata.

Si possono individuare vari tipi di dati da definire, direttamente derivati
dalle definizioni matematiche: processo nullo (\lstinline{nil}), azione di
input (\lstinline{input}), azione di output (\lstinline{output}), prefisso
(\lstinline{prefix}), restrizione (\lstinline{restriction}), replicazione
(\lstinline{replication}) e composizione parallela (\lstinline{composition}).
Le definizioni sono identiche a quelle di \lstinline{name}, e vengono quindi
tralasciate.

Oltre ai predicati definiti automaticamente dalla sintassi
\lstinline{struct}, sono stati definiti due predicati di supporto:

\begin{lstlisting}
      ; process? : any/c -> boolean?
      (define (process? v)
        (or
          (nil? v)
          (prefix? v)
          (restriction? v)
          (replication? v)
          (composition? v)))

      ; action? : any/c -> boolean?
      (define (action? v)
        (or
          (input? v)
          (output? v)))

\end{lstlisting}

Questi predicati riconoscono rispettivamente ogni tipo di processo e ogni
tipo di azione; verranno usati anche nella definizione di contratti.

Come si \`e visto nella sezione dedicata al $\pi$-calcolo, \`e necessario
che siano rispettate alcune regole durante la costruzione di processi: ad
esempio, i nomi oggetto di un'azione devono essere distinti tra loro.

Con delle guardie adeguate, \`e possibile assicurarsi che azioni non
rispettose di questo requisito non vengano mai create: per le azioni viene
utilizzata una semplice guardia, che si limita a verificare che i parametri
passati al costruttore siano rispettivamente un nome e una lista non vuota
di nomi distinti.

La guardia per la replicazione \`e altrettanto semplice: \`e sufficiente
infatti assicurarsi che il parametro passato al costruttore sia un processo
tramite l'uso del predicato \lstinline{process?}.

Leggermente pi\`u complessa \`e la guardia della restrizione. Quando ci si
trova a restringere un nome $x$ in un processo $P$ si deve garantire che
$x$ non sia presente tra i nomi legati di $P$, se si vuole che il processo
risultante rispetti la propriet\`a di avere tutti i nomi legati distinti.

La guardia della restrizione garantisce che la propriet\`a sia sempre
verificata:

\begin{lstlisting}
      (define (restriction-guard x p type-name)
        (let ([bound (process-bound-names p)])
          (values x
                  (if (not (set-member? bound x))
                      p
                      (process-refresh-name p x)))))
\end{lstlisting}

dove non \`e stato riportato il controllo iniziale sui tipi dei parametri,
il quale assicura che \lstinline{x} soddisfi \lstinline{name?} e
\lstinline{p} soddisfi \lstinline{process?}.

Tramite la sintassi \lstinline{let}, alla variabile \lstinline{bound} viene
assegnato l'insieme dei nomi legati del processo \lstinline{p}, ottenuto
con la chiamata \lstinline{(process-bound-names p)}; poi, se il nome
\lstinline{x} \`e tra questi, al posto di \lstinline{p} viene usato per la
costruzione della restrizione il processo restituito da
\lstinline{(process-refresh-name p x)}.

La definizione di \lstinline{process-bound-names}, cos\`i come quella delle
procedure ad essa strettamente collegate, \lstinline{process-free-names} e
\lstinline{process-names}, \`e direttamente derivata dalla teoria del
$\pi$-calcolo: nel caso dei nomi legati, ad esempio, si ha

\begin{lstlisting}
      ; process-bound-names : process? -> (setof name?)
      (define (process-bound-names process)
        (match process
          [(nil)             (bound-names/nil)]
          [(input x y)       (bound-names/input x y)]
          [(output x y)      (bound-names/output x y)]
          [(prefix a p)      (bound-names/prefix a p)]
          [(restriction x p) (bound-names/restriction x p)]
          [(replication p)   (bound-names/replication p)]
          [(composition p q) (bound-names/composition p q)]))
\end{lstlisting}

La procedura pubblica fa uso della sintassi \lstinline{match} per distinguere
il tipo di processo, e delega il calcolo dei nomi legati alle rispettive
procedure private specializzate:

\begin{lstlisting}
      ; Il processo nullo non ha nomi legati
      (define (bound-names/nil)
        (set))

      ; I nomi legati di sono quelli in oggetto
      (define (bound-names/input x y)
        (list->set y))

      ; Un'azione di output non ha nomi legati
      (define (bound-names/output x y)
        (set))

      ; I nomi legati in un'azione dipendono dal tipo di azione
      (define (bound-names/action a)
        (match a
          [(input x y)  (bound-names/input x y)]
          [(output x y) (bound-names/output x y)]))

      ; I nomi legati in un prefisso sono quelli legati
      ; dall'azione, uniti a quelli legati nella continuazione
      (define (bound-names/prefix a p)
        (set-union (process-bound-names a)
                   (process-bound-names p)))

      ; I nomi legati in una restrizione sono quelli legati nel
      ; processo sotto restrizione e il nome oggetto della restrizione
      (define (bound-names/restriction x p)
        (set-union (set x)
                   (process-bound-names p)))

      ; I nomi legati non vengono modificati dalla replicazione
      (define (bound-names/replication p)
        (process-bound-names p))

      ; I nomi legati in una composizione sono i nomi legati nel primo
      ; processo uniti a quelli legati nel secondo processo
      (define (bound-names/composition p q)
        (set-union (process-bound-names p)
                   (process-bound-names q)))
\end{lstlisting}

Si pu\`o notare che la procedura \lstinline{process-bound-names} accetta
come parametro anche un \lstinline{input} o un \lstinline{output}, che
non rispettano il contratto \lstinline{process?}: questo semplifica la
scrittura di alcune procedure interne al modulo, e non causa problemi
agli utilizzatori esterni perch\'e il contratto usato per esportarla,
\lstinline{(process? . -> . (setof name?))}, impedisce un utilizzo scorretto
all'esterno del modulo.

La procedura \lstinline{process-refresh-name} permette di eliminare tutti
i riferimenti al nome \lstinline{x} da un processo \lstinline{p}
mantenendone invariato il significato, evitando cio\`e la cattura di nomi:
effettuare questo tipo di sostituzione \`e fondamentale per garantire che
i nomi legati in un processo siano distinti da tutti gli altri nomi.

Prima di illustrare il funzionamento di \lstinline{process-refresh-name},
bisogna spiegare il criterio utilizzato per la generazione di nuovi nomi.

Si ricordi che tutti i nomi rispettano il contratto \lstinline{id-string?},
sono cio\`e costituiti da una serie di caratteri, eventualmente seguita
da una parte numerica. La procedura privata \lstinline{parts}, definita
all'interno del modulo \lstinline{pity/name}, estrae le due parti
utilizzando un'espressione regolare:

\begin{lstlisting}
      (define (parts n)
         (let* ([parts (regexp-match #rx"^([a-zA-Z]+)([0-9]*)$" n)]
                [str (cadr parts)]
                [num (caddr parts)])
           (cons str num)))
\end{lstlisting}

Per generare un nuovo nome, \`e sufficiente incrementare la parte numerica
del nome di partenza: \`e esattamente quello che fa la procedura pubblica
\lstinline{name-refresh}

\begin{lstlisting}
      ; name-refresh : name? -> name?
      (define (name-refresh self)
        (let* ([n (name-n self)]
               [parts (parts n)]
               [str (car parts)]
               [num (cdr parts)]
               [num (if (equal? num "") 0 (string->number num))]
               [num (+ num 1)]
               [n (string-append str (number->string num))])
          (name n)))
\end{lstlisting}

La prima volta, alla variabile \lstinline{num} viene assegnato il secondo
valore restituito da \lstinline{(parts n)}: questo valore \`e una stringa
contenente la parte numerica, che pu\`o essere vuota.

Nel secondo assegnamento viene usata la sintassi \lstinline{if} per
assegnare il valore numerico \lstinline{0} alla stringa vuota: come
conseguenza, un nome privo di parte numerica come quello ottenuto usando
\lstinline{(name "a")} viene considerato avere parte numerica pari a
\lstinline{0}. Subito dopo, la parte numerica \`e incrementata e un nuovo
nome viene restituito.

Questo algoritmo di generazione nomi ha il vantaggio di essere veloce e
semplice da implementare; inoltre, il nome generato \`e sempre prevedibile
conoscendo il nome di partenza, e questo rende possibile la verifica
dell'implementazione tramite casi di test, cosa che non sarebbe possibile
se ad esempio il nuovo nome venisse generato casualmente.

Due nomi possono essere confrontati tra di loro semplicemente confrontando
le stringhe che li rappresentano:

\begin{lstlisting}
      ; name-max : name? name? -> name?
      (define (name-max a b)
        (let ([na (name-n a)]
              [nb (name-n b)])
          (if (string>=? na nb)
              a
              b)))
\end{lstlisting}

Si noti che, con questa definizione di \lstinline{name-max}, si verifica

\begin{lstlisting}
      (name-max (name "x5") (name "y2")) ; ==> (name "y2")
\end{lstlisting}

il che potrebbe causare problemi in alcune applicazioni. Per capire quando
due nomi possano essere confrontati usando \lstinline{name-max} senza
ottenere risultati inaspettati, si pu\`o usare il predicato
\lstinline{name-compatible?}, definito come

\begin{lstlisting}
      ; name-compatible? : name? name? -> boolean?
      (define (name-compatible? a b)
        (let* ([n (name-n a)]
               [p (parts n)]
               [na (car p)]
               [n (name-n b)]
               [p (parts n)]
               [nb (car p)])
          (equal? na nb)))
\end{lstlisting}

che restituisce \lstinline{#t} se i due nomi differiscono solamente nella
parte numerica, \lstinline{#f} altrimenti.

Per implementare correttamente \lstinline{process-refresh-name}, tuttavia,
non \`e sufficiente che il nome designato sia sostituito con quello
restituito da \lstinline{name-refresh}: per capire come questo approccio
ingenuo potrebbe causare problemi, basti pensare che sostituire in questa
maniera il nome $x_1$ nel processo

\begin{pilisting}
$
    y(x_2) . \, \overline{x_1}\langle z\rangle . 0
$
\end{pilisting}

porta al processo

\begin{pilisting}
$
    y(x_2) . \, \overline{x_2}\langle z\rangle . 0
$
\end{pilisting}

che ha significato ben diverso da quello originale.

\`E necessario avere la capacit\`a di generare nomi che sono garantiti
essere nuovi, cio\`e non presenti nel processo corrente: per fare ci\`o, si
pu\`o usare la procedura privata \lstinline{fresh-name}

\begin{lstlisting}
      (define (fresh-name p n)
        (let* ([names (process-names p)]
               [m (name-refresh n)]
               [ok? (lambda (x)
                      (or (not (name-compatible? x m))
                          (equal? (name-max x m) m)))]
               [res (set-map names ok?)])
           (if (and (not (member #f res))
                    (not (set-member? names m)))
               m
               (fresh-name p m))))
\end{lstlisting}

Questa procedura trova un nome compatibile con \lstinline{n} che \`e
garantito non essere gi\`a presente in \lstinline{p}.

Il modo in cui lavora \`e concettualmente molto semplice: si limita a
chiamare \lstinline{name-refresh} finch\'e il nome ottenuto non \`e
un nome non presente in \lstinline{p}.

Per prima cosa, i nomi che hanno occorrenze in \lstinline{p} vengono
calcolati usando \lstinline{process-names}, poi il nome ottenuto chiamando
\lstinline{(name-refresh n)} viene assegnato alla variabile \lstinline{m}.

Si vuole trovare il nome compatibile con \lstinline{n}, non presente in
\lstinline{p}, con parte numerica maggiore: per verificare se
\lstinline{m} soddisfa queste condizioni, si dichiara un predicato di
supporto, \lstinline{ok?}, che accetta come parametro un nome e restituisce
\lstinline{#t} se tale nome \`e o non compatibile con \lstinline{m}, o ha
parte numerica minore.

Il predicato \lstinline{ok?} viene chiamato su tutti i nomi di \lstinline{p}
usando \lstinline{set-map}; la lista ottenuta, \lstinline{res}, contiene
una serie di valori di verit\`a.

Se questi non sono tutti uguali a \lstinline{#t}, vuol dire che tra i nomi
di \lstinline{p} almeno uno \`e compatibile con \lstinline{m} e ha parte
numerica maggiore: in tal caso, la funzione \lstinline{fresh-name} viene
chiamata nuovamente, questa volta con argomento \lstinline{m}, per tentare
il nome successivo.

Se invece tutti i confronti hanno avuto successo, viene eseguito un
controllo aggiuntivo per verificare che \lstinline{m} non sia gi\`a tra
i nomi di \lstinline{p} (condizione che si verifica ad esempio quando viene
chiesto di sostituire $x_1$ in $\overline{x_1}\langle x_2\rangle$), e il
nome \lstinline{m} viene restituito.

TODO process-refresh-name

La guardia per il prefisso \`e ancora pi\`u complessa:

\begin{lstlisting}
      (define (prefix-guard a p type-name)
        (let* ([refresh (lambda (q n)
                          (let* ([an (fresh-name a n)]
                                 [qn (fresh-name q n)]
                                 [nn (name-max an qn)])
                            (replace-name q n nn)))]
               [bound (process-bound-names p)]
               [bound (set-intersect bound (process-names a))]
               [bound (set->list bound)]
               [p (foldl (flip refresh) p bound)])
          (if (not (input? a))
              (values a p)
              (let* ([bound (process-bound-names a)]
                     [x (input-x a)]
                     [y (input-y a)]
                     [na (fresh-name a x)]
                     [np (fresh-name p x)]
                     [nn (name-max na np)]
                     [y (if (not (set-member? bound x))
                            y
                            (list-replace y x nn))]
                     [a (input x y)]
                     [p (if (not (set-member? bound x))
                            y
                            (replace-name p x nn))])
                (values a p)))))
\end{lstlisting}

dove ancora una volta sono stati tralasciati i controlli iniziali, che
assicurano \lstinline{a} rispetti il contratto \lstinline{action?} e
\lstinline{p} rispetti il contratto \lstinline{process?}.

Questa procedura si deve assicurare che due situazioni di errore vengano
gestite correttamente: la prima \`e quella che si verifica quando uno dei
nomi di \lstinline{a} \`e gi\`a presente tra i nomi legati di
\lstinline{p}, come ad esempio nel processo
$\overline{b}\langle c\rangle.a(b).0$, dove l'occorrenza di $b$
in $\overline{b}\langle c\rangle$ fa riferimento ad un nome che \`e diverso
da quello a cui fa riferimento l'occorrenza di $b$ in $a(b)$; una volta
effettuato il cambio di nomi si deve ottenere il processo
$\overline{b}\langle c\rangle.a(b_1).0$, privo di ambiguit\`a.

In questo primo caso \`e sufficiente rinominare l'occorrenza legata di $b$:
nella procedura \lstinline{prefix-guard}, si comincia con il dichiarare una
procedura di supporto, \lstinline{refresh}, che funziona nello stesso modo
di \lstinline{process-refresh-name}, ma si assicura che il nome usato per
la sostituzione non sia presente n\'e in \lstinline{p} n\'e in
\lstinline{a}, evitando quindi la cattura di nomi.

Definita questa procedura, si prosegue con l'assegnare alla variabile
\lstinline{bound} l'insieme dei nomi legati in \lstinline{p}, al quale
vengono sottratti tutti i nomi che non compaiono in \lstinline{a};
l'insieme viene poi convertito in una lista, e tramite la chiamata

\begin{lstlisting}
      (foldl (flip refresh) p bound)
\end{lstlisting}

ci sia assicura che i nomi trovati (che sono quelli problematici) siano
stati sostituiti da nomi non presenti n\'e in \lstinline{p} n\'e in
\lstinline{a}.

Se l'azione \`e di output, non serve controllare altro, e si pu\`o procedere
con la normale costruzione della struttura.

Se invece l'azione \`e di input, bisogna gestire la seconda situazione
d'errore, ovvero quella che si verifica in un processo quale
$a(a).\overline{a}\langle b\rangle.0$, dove la prima istanza di $a$ si
riferisce ad un nome diverso da quello a cui fanno riferimento le altre due
occorrenze, quindi \`e necessario operare un cambio di nome per
differenziarla, ottenendo il processo
$a(a_1).\overline{a_1}\langle b\rangle.0$

Anche in questo caso, si comincia con il trovare i nomi legati dall'azione
\lstinline{a}, e si usa la procedura \lstinline{fresh-name} per ottenere
una nuova versione del nome \lstinline{x} usato per la trasmissione: se il
nome in questione fa parte dei nomi legati da \lstinline{a}, viene
operata la sua sostituzione all'interno della lista \lstinline{y} di nomi
ricevuti.

Per quanto riguarda il processo \lstinline{p}, \`e sufficiente operare una
sostituzione per mezzo della procedura \lstinline{replace-name}.
