\subsection{Processi}

Il modulo \lstinline{pity/process} contiene l'implementazione dei processi,
che costituisce la parte pi\`u importante, oltre che pi\`u complessa, della
libreria sviluppata.

Si possono individuare vari tipi di dati da definire, direttamente derivati
dalle definizioni matematiche: processo nullo (\lstinline{nil}), azione di
input (\lstinline{input}), azione di output (\lstinline{output}), prefisso
(\lstinline{prefix}), restrizione (\lstinline{restriction}), replicazione
(\lstinline{replication}) e composizione parallela (\lstinline{composition}).
Le definizioni sono identiche a quelle di \lstinline{name}, e vengono quindi
tralasciate.

Oltre ai predicati definiti automaticamente dalla sintassi
\lstinline{struct}, sono stati definiti due predicati di supporto:

\begin{lstlisting}
      ; process? : any/c -> boolean?
      (define (process? v)
        (or
          (nil? v)
          (prefix? v)
          (restriction? v)
          (replication? v)
          (composition? v)))

      ; action? : any/c -> boolean?
      (define (action? v)
        (or
          (input? v)
          (output? v)))

\end{lstlisting}

Questi predicati riconoscono rispettivamente ogni tipo di processo e ogni
tipo di azione; verranno usati anche nella definizione di contratti.

Come si \`e visto nella sezione dedicata al $\pi$-calcolo, \`e necessario
che siano rispettate alcune regole durante la costruzione di processi: ad
esempio, i nomi oggetto di un'azione devono essere distinti tra loro.

Con delle guardie adeguate, \`e possibile assicurarsi che azioni non
rispettose di questo requisito non vengano mai create:

\begin{lstlisting}
      (define (action-guard x y type-name)
        (when (not (name? x))
              (error type-name
                     "x is not a name?"))
        (when (not ((non-empty-listof-distinct name?) y))
              (error type-name
                     "y is not a (non-empty-listof-distinct name?)"))
        (values x y))
\end{lstlisting}

dove \lstinline{(non-empty-listof-distinct name?)} restituisce un contratto
soddisfatto da liste di \lstinline{name} non vuote e prive di duplicati.
