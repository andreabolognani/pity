\subsection{Generazione di nuovi nomi}

Lavorando con i processi, ci si trover\`a spesso a dover generare dei
nuovi nomi per garantire che le propriet\`a richieste per dal type
system siano rispettate.

Un approccio possibile sarebbe quello di generare un nuovo nome creando
una stringa casuale, ma tale approccio non \`e desiderabile per due
motivi: il primo \`e che richiede potenzialmente molti passaggi, dato
che non esiste garanzia che il nome generato casualmente non sia uguale
ad un nome gi\`a in uso; il secondo \`e che renderebbe impossibile la
scrittura di una test suite automatizzata, rendendo di conseguenza il
testing del software allo stesso tempo pi\`u complesso e meno affidabile.

Si \`e quindi seguito un approccio alternativo: si ricordi che tutti i
nomi rispettano il contratto \lstinline{id-string?}, sono cio\`e
costituiti da una serie di caratteri, eventualmente seguita da una parte
numerica. La procedura privata \lstinline{parts} estrae le due parti
utilizzando un'espressione regolare:

\begin{lstlisting}
      (define (parts n)
         (let* ([parts (regexp-match #rx"^([a-zA-Z]+)([0-9]*)$" n)]
                [str (cadr parts)]
                [num (caddr parts)])
           (cons str num)))
\end{lstlisting}

Per generare un nuovo nome, \`e sufficiente incrementare la parte numerica
del nome di partenza: \`e esattamente quello che fa la procedura pubblica
\lstinline{name-refresh}

\begin{lstlisting}
      ; name-refresh : name? -> name?
      (define (name-refresh self)
        (let* ([n (name-n self)]
               [parts (parts n)]
               [str (car parts)]
               [num (cdr parts)]
               [num (if (equal? num "") 0 (string->number num))]
               [num (+ num 1)]
               [n (string-append str (number->string num))])
          (name n)))
\end{lstlisting}

La prima volta, alla variabile \lstinline{num} viene assegnato il secondo
valore restituito da \lstinline{(parts n)}: questo valore \`e una stringa
contenente la parte numerica, che pu\`o essere vuota.

Nel secondo assegnamento viene usata la sintassi \lstinline{if} per
assegnare il valore numerico \lstinline{0} alla stringa vuota: come
conseguenza, un nome privo di parte numerica come quello ottenuto usando
\lstinline{(name "a")} viene considerato avere parte numerica pari a
\lstinline{0}. Subito dopo, la parte numerica \`e incrementata e un nuovo
nome viene restituito.

Due nomi possono essere confrontati tra di loro semplicemente confrontando
le stringhe che li rappresentano:

\begin{lstlisting}
      ; name-max : name? name? -> name?
      (define (name-max a b)
        (let ([na (name-n a)]
              [nb (name-n b)])
          (if (string>=? na nb)
              a
              b)))
\end{lstlisting}

Si noti che, con questa definizione di \lstinline{name-max}, si verifica

\begin{lstlisting}
      (name-max (name "x5") (name "y2")) ; ==> (name "y2")
\end{lstlisting}

il che potrebbe causare problemi in alcune applicazioni. Per capire quando
due nomi possano essere confrontati usando \lstinline{name-max} senza
ottenere risultati inaspettati, si pu\`o usare il predicato
\lstinline{name-compatible?}, definito come

\begin{lstlisting}
      ; name-compatible? : name? name? -> boolean?
      (define (name-compatible? a b)
        (let* ([n (name-n a)]
               [p (parts n)]
               [na (car p)]
               [n (name-n b)]
               [p (parts n)]
               [nb (car p)])
          (equal? na nb)))
\end{lstlisting}

che restituisce \lstinline{#t} se i due nomi differiscono solamente nella
parte numerica, \lstinline{#f} altrimenti.
