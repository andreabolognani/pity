\documentclass[a4paper]{article}

\usepackage{listings}
\lstdefinelanguage{Racket}
	{morekeywords={lambda, define, define-struct, let, let*, list},
     sensitive=true,
     morecomment=[l]{;},
     morestring=[b]{"},}
\lstset{language=Racket,
        basicstyle=\ttfamily\small,
        keywordstyle=\bf,
        emphstyle=\underbar,
        showstringspaces=false,
        numbers=left,
        numberstyle=\tiny,
        numbersep=10pt}


\begin{document}

\title{Pity - Pi-Calculus Type Checking}
\author{Andrea Bolognani}
\date{Ottobre 2010}

\maketitle


\begin{abstract}
Contenuto della tesi, spiegato molto brevemente e molto ad alto livello.
\end{abstract}


\clearpage

\tableofcontents

\clearpage


\section{Introduzione}

Introduzione al lavoro svolto. Riassume tutto il lavoro svolto, quindi
va scritta come ultima cosa.


\section{Stato dell'arte}

Punto di partenza del lavoro: breve introduzione al Pi-Calcolo, breve
introduzione al linguaggio di programmazione Scheme e alla sua
implementazione Racket.


\section{Sviluppo}

Uno o pi\`u capitoli dedicati al corpo vero e proprio della tesi.

Vediamo ad esempio un pezzetto di codice Racket:

\lstset{emph={double}}
\begin{lstlisting}
; Returns the double of a number
(define double
  (lambda (x) (* 2 x)))

(display "Two times three is ")
(display (double 3))
\end{lstlisting}

In questo caso, la funzione \lstinline{double} era definita nel pezzo
di codice, quindi il suo identificativo era sottolineato.


\section{Conclusioni}

Simile all'introduzione, riassume il lavoro svolto e spiega in che
modo si potrebbe andare avanti a sviluppare il lavoro.

\end{document}
