\documentclass[a4paper,draft]{article}

\begin{document}

\title{Pity - Pi-Calculus Type Checking}
\author{Andrea Bolognani}
\date{Ottobre 2010}

\maketitle


\begin{abstract}
Contenuto della tesi, spiegato molto brevemente e molto ad alto livello.
\end{abstract}


\clearpage

\tableofcontents

\clearpage


\section{Introduzione}

Introduzione al lavoro svolto. Riassume tutto il lavoro svolto, quindi
va scritta come ultima cosa.


\section{Stato dell'arte}

Punto di partenza del lavoro: breve introduzione al Pi-Calcolo, breve
introduzione al linguaggio di programmazione Scheme.


\section{Sviluppo}

Uno o pi\`u capitoli dedicati al corpo vero e proprio della tesi.


\section{Conclusioni}

Simile all'introduzione, riassume il lavoro svolto e indica in quali
modi si potrebbe andare avanti a sviluppare il lavoro.

\end{document}
